%%%%%%%%%%%%%%%%%%%%%%%%%%%%%%%%%%%%%%%%%%%%%%%%%%%%%%%%%%%%%%%%%%%%%%%%%%%
%%%                                                                     %%%
%%%   LaTeX template voor het verslag van P&O: Computerwetenschappen.   %%%
%%%                                                                     %%%
%%%   Opties:                                                           %%%
%%%     tt1     Tussentijdsverslag 1                                    %%%
%%%     tt2     Tussentijdsverslag 2                                    %%%
%%%     tt3     Tussentijdsverslag 3                                    %%%
%%%     eind    Eindverslag                                             %%%
%%%                                                                     %%%
%%%   2 oktober 2012                                                    %%%
%%%   Versie 1.0                                                        %%%
%%%                                                                     %%%
%%%%%%%%%%%%%%%%%%%%%%%%%%%%%%%%%%%%%%%%%%%%%%%%%%%%%%%%%%%%%%%%%%%%%%%%%%%

\documentclass[tt1]{penoverslag}

%%% PACKAGES
\usepackage{lipsum}
\usepackage{hyperref}


\begin{document}

\team{Brons} % teamkleur
\members{Mattias Buelens\\
		 Vital D'haveloose\\
		 Dennis Frett\\
		 Stijn Hoskens\\
		 Matthias Moulin} % teamleden

\maketitlepage

\begin{abstract}
\lipsum[1-2]
\end{abstract}

\tableofcontents

\newpage

% == INLEIDING == %
\section{Inleiding}
De bedoeling van deze deelopdracht is allereerst het bouwen van een fysieke robot door middel van Lego Mindstorms.
Deze robot moest vervolgens zo geprogrammeerd worden, dat deze bestuurd kon worden met de pijltjestoetsen van de computer.
Ook moest hij in staat zijn een regelmatige veelhoek af te rijden met gegeven lengte en aantal hoeken.
Een simulator van deze robot zou deze veelhoek ook kunnen rijden.
De besturing van de fysieke en virtuele robot verloopt volgens een duidelijke GUI, die tevens ook berichten en eventuele debug informatie uitwisselt.


\section{Bouw robot}
Voor de bouw van de robot hebben we gekozen voor de Express-Bot zoals beschreven is op \url{http://www.nxtprograms.com/9797/express-bot/pdf/ExpressBot.pdf}. 
Enkele voordelen van de gekozen constructie zijn: de accu is eenvoudig loskoppelbaar, de grote functionaliteit, precies draaien is mogelijk doordat hij gewoon rond zijn eigen as kan draaien. De robot is ook eenvoudig uitbereidbaar, en deze uitbereidingen staan ook beschreven in de instructies, zoals daar zijn: de verschillende sensors, de ``Pivoting Head Explorer'',...

\subsection{Fysieke bouw}
\begin{itemize}
\item De robot bestaat uit twee grote wielen, aangedreven door de motoren, die aan weerskanten van de NXT Brick worden geplaatst. De stabilisatie van de gehele constructie gebeurt door een derde wiel aan de achterkant, dat vrij kan roteren.
\item We hebben voor deze constructie gekozen nadat we volgende voorstellen hebben afgewezen: Het standaard-model hebben we bijna onmiddellijk afgevoerd, wegens te klein, en daarom misschien niet uitbreidbaar genoeg. Ook hadden we ons oog laten vallen op een model met vier wielen die door middel van een derde servo-motor op de twee voorste wielen zichzelf bestuurde. Hierdoor had deze robot echter een te grote draaicirkel nodig.
\end{itemize}

\subsection{Meetresultaten}
De nauwkeurigheid van de robot wordt beoordeeld op basis van drie criteria: kromming wanneer hij  rechtdoor moet rijden, correctheid van gereden afstand, correctheid van geroteerde hoek.

\subsubsection{Rechtdoor rijden}
De robot had bij het begin moeilijkheden met het rechtdoor rijden, dit bleek aan de banden te liggen. Nadat we de grotere wielen vervingen door kleinere exemplaren, met dunnere banden, bleek het rechtdoor rijden beter te lukken. De afwijking die er nog was, was enkel te verklaren door een fout op de motor. We hebben dit kunnen oplossen door het linkerwiel een andere diameter te geven in de software dan het rechterwiel. Door veelvuldig te testen hebben we het optimale verschil tussen de diameters kunnen bepalen. In een finaal experiment werd de fout op een recht pad gemeten. De robot reed hierbij een afstand van 2.4m, op vier equidistante punten op dit pad werd de loodrechte afstand van de robot tot een rechte gemeten. %TODO: Grafieken! %

\subsubsection{Gereden afstand}
Om na te gaan of de afstand die de robot rijdt correct is, moest de robot een op voorhand-afgemeten afstand rijden. Aanvankelijk reed hij met een kleine afwijking. Door beide diameters van de wielen een klein beetje aan te passen in de software, waarbij we het verschil tussen de twee diameters gelijk hielden, reed hij uiteindelijk de gevraagde afstand correct.

\subsubsection{Rotatie}
Om een beeld te krijgen van hoe correct de robot roteerde, lieten we deze telkens 360 graden draaien en keken we of deze terug in dezelfde positie terecht kwam. Door kleine aanpassingen te maken in de breedte-parameter van de robot, kregen we betere resultaten. Als volgende test keken we na of de robot correct een vierkant met zijden 1m kon beschrijven. De afwijking hier op was iets groter dan bij de vorige test. Dit kwam doordat er nu ook gereden moest worden. Na elke rotatie moest de robot opnieuw accelereren, hierdoor kwam er telkens een kleine fout op de hoek. Door preciezer de breedte-parameter in te stellen, konden we deze afwijking minimaliseren.%TODO: Grafieken! %

\subsection{\ldots}
\ldots


% == ALGORITMES == %
\section{Algoritmes}
\lipsum[4]

\subsection{Rechtzetten robot op witte lijn}
\begin{itemize}
\item Beschrijving van het algoritme voor het rechtzetten van de robot op de lijn.
\end{itemize}

\subsection{Lezen barcodes}
\begin{itemize}
\item Beschrijving van het algoritme om de robot een barcode te laten lezen.
\end{itemize}

\subsection{Sturing van de robot}
\begin{itemize}
\item Beschrijving van het algoritme dat de robot stuurt.
\end{itemize}

\subsection{Navigatie door doolhof}
\begin{itemize}
\item Beschrijving van het algoritme dat selecteert welk navigatiesysteem gebruikt wordt om de robot het doolhof te laten verkennen.
\end{itemize}

\subsection{\ldots}
\ldots


% == SOFTWARE == %
\section{Software}
\lipsum[5]

\subsection{Bluetooth}
\ldots

\subsection{GUI}
\begin{itemize}
\item Geef hier een duidelijk beeld van het design van de user interface (welke beslissingen, waarom?, enz.) en een overzicht van de functionaliteiten.
\end{itemize}

\subsection{Simulator}
\begin{itemize}
\item Beschrijving van de simulator.
\end{itemize}

\subsection{Software design}
\begin{itemize}
\item Geef hier een klassediagramma en een overzicht van de verschillende methodes.
\end{itemize}

\subsection{\ldots}
\ldots


% == BESLUIT == %
\section{Besluit}
\lipsum[6-7]



\newpage
\makeappendix

\section{Demo 1}

\subsection{Resultaten}
\ldots

\subsection{Conclusies}
\ldots

\subsection{Oplijsting aanpassingen verslag}
Hier komt een summiere weergave van welke secties uit het vorige verslag gewijzigd werden.


\section{Demo 2}

\subsection{Resultaten}
\ldots

\subsection{Conclusies}
\ldots

\subsection{Oplijsting aanpassingen verslag}
Hier komt een summiere weergave van welke secties uit het vorige verslag gewijzigd werden.


\section{Demo 3}

\subsection{Resultaten}
\ldots

\subsection{Conclusies}
\ldots

\subsection{Oplijsting aanpassingen verslag}
Hier komt een summiere weergave van welke secties uit het vorige verslag gewijzigd werden.


\section{Beschrijving van het proces}
\begin{itemize}
\item Welke moeilijkheden heb je ondervonden tijdens de uitwerking?
\item Welke lessen heb je getrokken uit de manier waarop je het project hebt aangepakt?
\item Hoe verliep het werken in team? Op welke manier werd de teamco\"ordinatie en planning aangepakt?
\end{itemize}


\section{Beschrijving van de werkverdeling}
\begin{itemize}
\item Geef voor elk van de groepsleden aan aan welke delen ze hebben meegewerkt en welke andere taken ze op zich hebben genomen.
\item Rapporteer in tabelvorm hoeveel uur elk groepslid elke week aan het project gewerkt heeft, zowel tijdens als buiten de begeleide sessies. Geef ook totalen per groepslid voor het volledige semester.
\end{itemize}


\section{Kritisch analyse}
\begin{itemize}
\item Maak een analyse van de sterke en zwakke punten van het project. Welke punten zijn vatbaar voor verbetering. Wat zou je, met je huidige kennis, anders aangepakt hebben?
\end{itemize}



\newpage
\bibliographystyle{siam}
\bibliography{biblio.bib}


\end{document}
